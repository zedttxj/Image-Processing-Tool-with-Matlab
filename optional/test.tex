
%-------Packages---------
\documentclass[11pt]{amsart}
\usepackage{amsmath}
\usepackage{latexsym,amssymb,verbatim}
%\usepackage{xypic}
\usepackage{shuffle} % Used for the shuffle product symbol.
\usepackage{caption}
\usepackage{graphicx}
\usepackage{amsfonts}
\usepackage[all,arc]{xy}
\usepackage{enumerate}
\usepackage{mathrsfs}
\usepackage{setspace}
\usepackage{color}
\usepackage{xcolor}
\usepackage{color, colortbl}
\usepackage{pdfpages}
\usepackage{tikz}
\usetikzlibrary{arrows}
\usepackage{enumitem}
\usepackage{stmaryrd}
\usepackage{extarrows}
\usepackage{array}
\usepackage{young}
\usepackage{youngtab}
\usepackage{ytableau}
\definecolor{mygreen}{rgb}{0,.4,0}
\definecolor{myblue}{rgb}{0,0,.5}
\usepackage[bookmarks=true,colorlinks,linkcolor=myblue,citecolor=mygreen]{hyperref}
\usepackage{bbding}
\usepackage[margin=1in]{geometry}
\usetikzlibrary{calc,decorations.pathreplacing,calligraphy,matrix}
\usepackage{hyperref}
%%%%%%%%%%%%%%%%%%%%%%%%%%%%%%%%%%%%%%%%%%%%%%%%%%%%%%
%%%%%%%%%%%%%%%%%%%%%%%%%%%%%%%%%%%%%%%%%%%%%%%%%%%%%%
\tikzset{Dotted/.style={% https://tex.stackexchange.com/a/52856/194703
		line width=\pgfkeysvalueof{/tikz/Young/dot size},
		dash pattern=on 0.001\pgflinewidth off #1,line cap=round,
		shorten <=#1},Dotted/.default=3pt,
	vdots/.style={draw=none,path picture={
			\draw let \p1=(path picture bounding box.north),
			\p2=(path picture bounding box.south) in
			[Dotted={(\y1-\y2)/4}]
			(\p1) -- (\p2);
	}},
	cdots/.style={draw=none,path picture={
			\draw let \p1=(path picture bounding box.east),
			\p2=(path picture bounding box.west) in
			[Dotted={(\x1-\x2)/4}]
			(\p2) -- (\p1);
	}},
	Young tableau/.style={matrix of math nodes,nodes in empty cells,
		nodes={draw,minimum size=\pgfkeysvalueof{/tikz/Young/cell size},inner sep=0.5pt},
		column sep=-\pgflinewidth,row sep=-\pgflinewidth},
	Young/.cd,cell size/.initial=1.75em,
	dot size/.initial=1.2pt
}

%------------------Theorems

%regular setup
%\setlength{\textheight}{225mm} 
%\setlength{\textwidth}{165mm} 
%\setlength{\oddsidemargin}{-0.2cm}
%\setlength{\evensidemargin}{-0.2cm}
%\linespread{1.1}

%small setup
\setlength{\textheight}{225mm} 
\setlength{\topmargin}{0.16cm}
\setlength{\textwidth}{165mm} 
\setlength{\oddsidemargin}{-0.2cm}
\setlength{\evensidemargin}{-0.2cm}



%very small 
%\setlength{\textheight}{235mm} 
%\setlength{\oddsidemargin}{-0.2cm}
%\setlength{\evensidemargin}{-0.2cm}
%\textwidth165mm
%\textheight23.4cm
%%%%%%%%%%%%%%%5%\hoffset-24mm
%\voffset-7mm

%\def\checkmark{\tikz\fill[scale=0.4](0,.35) -- (.25,0) -- (1,.7) -- (.25,.15) -- cycle;} 
%\checkmark 
\setlength{\parindent}{0pt}
\newtheorem{theorem}{Theorem}[section]
\newtheorem{proposition}[theorem]{Proposition}
\newtheorem{lemma}[theorem]{Lemma}
\newtheorem{corollary}[theorem]{Corollary}
%\theoremstyle{definition}
\newtheorem{definition}[theorem]{Definition}
\newtheorem{definitions}[theorem]{Definitions}
\newtheorem{example}[theorem]{Example}
\newtheorem{problem}[theorem]{Problem}
\theoremstyle{remark}
\newtheorem{remark}[theorem]{Remark}
\newtheorem{exercise}[theorem]{Exercise}
\newtheorem{remarks}[theorem]{Remarks}
\newtheorem{question}[theorem]{Question}
\newtheorem{claim}[theorem]{\bf\hspace*{-1ex}}
\renewenvironment{proof}{{\noindent\bf Proof.}}{\hfill $\Box$\par\vskip3mm}

\newcommand{\Ker}{{\rm Ker}\,}
\newcommand{\Coker}{{\rm Coker}\,}
\newcommand{\im}{{\rm Im}\,}
\newcommand{\coim}{{\rm Coim}\,}
\newcommand{\Hom}{{\rm Hom}}
\newcommand{\End}{{\rm End}}
\newcommand{\Ext}{{\rm Ext}}
\newcommand{\Mor}{{\rm Mor}\,}
\newcommand{\Aut}{{\rm Aut}\,}

\newcommand\Symm{\mathfrak{S}} % Symmetric group.
\newcommand\Lieg{\mathfrak{g}}
\newcommand\Liep{\mathfrak{p}}
\newcommand\ooo{\mathfrak{o}}
\newcommand\mmm{\mathfrak{m}}

\newcommand\kk{\mathbf{k}}
\newcommand\rr{\mathbf{r}}
\newcommand\kkk{\mathbf{K}}

\newcommand\starstar{{\boxplus}}

\newcommand\ii{{\mathbf{i}}}
\newcommand\jj{{\mathbf{j}}}
\newcommand\xx{{\mathbf{x}}}
\newcommand\XX{{\mathbf{X}}}
\newcommand\yy{{\mathbf{y}}}
\newcommand\YY{{\mathbf{Y}}}
\newcommand\zz{{\mathbf{z}}}
\newcommand\RRR{{\mathbf{R}}}
\newcommand\pt{\mathbf{pt}}

\newcommand\aaa{{\mathfrak{A}}}
\newcommand\sss{{\mathfrak{s}}}
\newcommand\bbb{{\mathfrak{B}}}
\newcommand\ccc{{\mathfrak{C}}}
\newcommand\ddd{{\mathfrak{D}}}
\newcommand\eee{{\mathfrak{E}}}
\newcommand\ppp{{\mathfrak{p}}}
\newcommand\PBT{{\mathcal{PBT}}}
\newcommand\AAA{{\mathcal{A}}}
\newcommand\BBB{{\mathcal{B}}}
\newcommand\CCC{{\mathcal{C}}}
\newcommand\FFF{{\mathcal{F}}}
\newcommand\GGG{{\mathcal{G}}}
\newcommand\HHH{{\mathcal{H}}}
\newcommand\JJJ{{\mathcal{J}}}
\newcommand\III{{\mathcal{I}}}
\newcommand\LLL{\mathcal{L}}
\newcommand\YYY{\mathcal{YD}}
\newcommand\DDD{\mathcal{BYD}}
\newcommand\MMM{{\mathcal{M}}}
\newcommand\PPP{{\mathcal{P}}}
\newcommand\BB{{\mathcal{B}}}
\newcommand\SSS{{\mathcal{S}}}
\newcommand\TTT{{\mathcal{T}}}

\newcommand\Aa{{\mathscr{A}}}
\newcommand\Bb{{\mathscr{B}}}
\newcommand\Cc{{\mathscr{C}}}
\newcommand\Pp{{\mathscr{P}}}
\newcommand\Qq{{\mathscr{Q}}}
\newcommand\Ee{{\mathscr{E}}}
\newcommand\Uu{{\mathscr{U}}}
\newcommand\Gg{{\mathscr{G}}}
\newcommand\Hh{{\mathscr{H}}}
\newcommand\Jj{{\mathscr{J}}}
\newcommand\Kk{{\mathscr{K}}}
\newcommand\Mm{{\mathscr{M}}}
\newcommand\Nn{{\mathscr{N}}}

\newcommand\ZZ{{\mathbb Z}}
\newcommand\FF{{\mathbb F}}
\newcommand\QQ{{\mathbb Q}}
\newcommand\NN{{\mathbb N}}
\newcommand\CC{{\mathbb C}}
\newcommand\RR{{\mathbb R}}
\newcommand\KK{{\mathbb K}}
\newcommand\LL{{\mathbb L}}
\newcommand\Proj{{\mathbb P}}
\providecommand{\tightlist}{%
  \setlength{\itemsep}{0pt}\setlength{\parskip}{0pt}}


%-----------------------------------------------------------

%--------Theorem Environments--------
%theoremstyle{plain} --- default
\newtheorem{thm}{Theorem}
\newtheorem{cor}[thm]{Corollary}
\newtheorem{prop}[thm]{Proposition}
\newtheorem{lem}[thm]{Lemma}
\newtheorem{conj}[thm]{Conjecture}
\newtheorem{quest}[thm]{Question}

\theoremstyle{definition}
\newtheorem{defn}[thm]{Definition}
\newtheorem{defns}[thm]{Definitions}
\newtheorem{con}[thm]{Construction}
\newtheorem{exmp}[thm]{Example}
\newtheorem{exmps}[thm]{Examples}
\newtheorem{notn}[thm]{Notation}
\newtheorem{notns}[thm]{Notations}
\newtheorem{addm}[thm]{Addendum}
\newtheorem{exer}[thm]{Exercise}

\theoremstyle{remark}
\newtheorem{rem}[thm]{Remark}
\newtheorem{rems}[thm]{Remarks}
\newtheorem{warn}[thm]{Warning}
\newtheorem{sch}[thm]{Scholium}

\setcounter{MaxMatrixCols}{20}



\newcommand\qtbin[2]{\left[\begin{matrix} #1 \\ #2 \end{matrix} \right]}
\newcommand\qbin[3]{\left[\begin{matrix} #1 \\ #2 \end{matrix} \right]_{#3}}
\newcommand\shuf[2]{{#1}\, \shuffle \,{#2}}

\newcommand\ncps[2]{{#1}\left<\left< {#2} \right>\right>}
\newcommand\powser[2]{{#1}\left[\left[ {#2} \right]\right]}

\makeatletter
\let\c@equation\c@thm
\makeatother
\numberwithin{equation}{section}

\bibliographystyle{plain}


\newcommand{\defqed}{\hspace*{\fill} $\square$}
\DeclareMathOperator{\lcm}{lcm}
% define the title

\title{Partition Differential Equations and Some Combinatorial Algebraic Structures}
\author{ADNAN Hashim ABDULWAHID}
\date{}

\begin{document}
	
	
	
	\begin{abstract}
		\noindent 
		I will do the abstract later.
		
	\end{abstract}
	
	\thanks{2020 \textit{Mathematics Subject Classifications}.  05E05, 05E40, 	05E16,  05E15, 16T15 }%, 20N99, 05E10} 
%Primary 16W30; Secondary 16S90, 16Lxx, 16Nxx, 18E40}
%\thanks{$^*$}
\date{}
\keywords{partitions, derivative, integral, symmetric functions, coalgebra, Stirling}

\maketitle



% generates the title



\noindent

\section{\textbf{Introduction and Preliminaries}}\label{intro.sec} 








\vspace{1cm}



\begin{theorem} \label{thm.dial.14}\textbf{}
	\begin{enumerate}[label=(\roman*)]
		\item The triple $(\Lambda, \eta, u$) is a commutative $\kk$-algebra, where the multiplication is the map	
		$$\Lambda \otimes \Lambda  \overset{\eta}{\longrightarrow} \Lambda, \,\, m_{\mu} \otimes m_{\nu} \mapsto m_{\mu \boxplus \nu},$$ 
		and the unit is the inclusion map $$\kk = \Lambda_0 \overset{u}{\longrightarrow} \Lambda.$$
		\item The triple $(\Lambda, \Theta, \epsilon$) is a $\kk$-coalgebra, where the comultiplication is the map	
		$$\Lambda  \overset{\Theta}{\longrightarrow} \Lambda \otimes \Lambda, \,\, m_{\lambda} \mapsto \sum\limits_{\substack{ (\mu, \nu) \in Par \times Par:\\ \mu \boxplus \nu = \lambda} } m_\mu \otimes m_\nu, $$ 
		and the  counit is the $\kk$-linear map $$\Lambda \overset{\epsilon}{\longrightarrow} \Lambda_0=\kk$$ with $\epsilon|_{\Lambda_0 = \kk} = id_{\kk}$  and  $\epsilon|_{I=\bigoplus_{n > 0} \Lambda_n} = 0$.  
	\end{enumerate}
\end{theorem}


\begin{definition}
	content...
\end{definition}





%%%%%%%%%%%%%%%%%%%%%%%%%%%%%%%%%%%%%%%%%%%%%%%%%%%%%%%%%%%%%%%%%%%%%




%%%%%%%%%%%%%%%%%%%%%%%%%%%%%%%%%%%%%%%%%%%%%%%%%%%%%%%%%%%%%%%%%%%%%









%%%%%%%%%%%%%%%%%%%%%%%%%%%%%%%%%%%%%%%%%%%%%%%%%%%%%%%%%%%%%%%%%%%


\vspace{1cm}








\begin{remark}
If $d=x_0=0$ and $M=1$, then $M_1=0$ and $y=f_\lambda(x)= \sum_{i=1}^{k}  m_i x^{i}$.
\end{remark}


Let 
\begin{align*}
y^{(d)}	& \,\, = M f_\lambda(x)= M \sum_{i=1}^{k}  m_i x^{i},
\end{align*}	
\begin{equation*}
(*)   \,\,\, . \,\,\, . \,\,\, . \,\,\, . \,\,\, . \,\,\, . \,\,\, . \,\,\,
\begin{cases}
	y^{(d-1)}(x_0)	& \,\, = M_{d-1} \\
	y^{(d-2)}(x_0)	& \,\, =  M_{d-2}\\
	\quad \quad .	& \quad \quad . \\
	\quad \quad .	& \quad \quad . \\
	\quad \quad .	& \quad \quad . \\
	y^{(1)}(x_0)	& \,\, =  M_1\\
	y^{(0)}(x_0)	& \,\, = M_0 
\end{cases}       
\end{equation*} be  a  $(\lambda,(*))$-partition differential equation.
We have 
\begin{align*}
y^{(d-1)}	& \,\, =  \int M \sum_{i=1}^{k}  m_i x^{i} dx\\
& \,\, = M \int  \sum_{i=1}^{k}  m_i x^{i} dx\\
& \,\, = M  \sum_{i=1}^{k}  m_i \int x^{i} dx\\
& \,\, = M  \sum_{i=1}^{k}  m_i   \frac{x^{i+1}}{i+1} + C_{d-1}\\
& \,\, = M   \sum_{i=1}^{k}  m_i   \frac{x^{i+1}}{(i+1)!/i!} + C_{d-1}\\
\end{align*}	

\begin{align*}
y^{(d-2)}	& \,\, =  \int y^{(d-1)} dx\\
& \,\, = M  \int \sum_{i=1}^{k}  m_i   \frac{x^{i+1}}{(i+1)!/i!} dx + \int C_{d-1} dx\\
& \,\, = M   \sum_{i=1}^{k}    \frac{m_i}{(i+1)!/i!} \int x^{i+1} dx + \int C_{d-1} dx\\
& \,\, = M   \sum_{i=1}^{k}  m_i   \frac{x^{i+2}}{(i+2)(i+1)} + C_{d-1} x+ C_{d-2}\\
& \,\, = M   \sum_{i=1}^{k}  m_i   \frac{x^{i+2}}{(i+2)!/i!} + C_{d-1}x+C_{d-2}\\
\end{align*}	


\begin{align*}
y^{(d-3)}	& \,\, =  \int y^{(d-2)} dx\\
& \,\, = M   \int  \sum_{i=1}^{k}  m_i   \frac{x^{i+2}}{(i+2)!/i!} dx + \int C_{d-1} x dx + \int C_{d-2} dx\\
& \,\, = M  \sum_{i=1}^{k}    \frac{m_i}{(i+2)!/i!} \int x^{i+1} dx + \int C_{d-1} x dx + \int C_{d-2} dx\\
& \,\, = M   \sum_{i=1}^{k}  m_i   \frac{x^{i+3}}{(i+3)(i+2)(i+1)} + C_{d-1} \frac{x^2}{2!} + C_{d-2}x + C_{d-3}\\
& \,\, = M   \sum_{i=1}^{k}  m_i   \frac{x^{i+3}}{(i+3)!/i!} + C_{d-1} \frac{x^2}{2!} + C_{d-2}\frac{x}{1!} + C_{d-3}\\
\end{align*}	


\begin{align*}
y^{(1)}	& \,\, =  \int y^{(2)} dx\\
& \,\, = M  \int  (\sum_{i=1}^{k}  m_i   \frac{x^{i+d-2}}{(i+d-2)!/i!} +  C_{d-1} \frac{x^{d-3}}{(d-3)!}+C_{d-2} \frac{x^{d-4}}{(d-4)!} + \cdots + C_3 \frac{x}{1!}+ C_2) dx\\
& \,\, = M   \sum_{i=1}^{k}   \int  m_i  \frac{x^{i+d-2}}{(i+d-2)!/i!} dx + \int C_{d-1} \frac{x^{d-3}}{(d-3)!} dx + \int C_{d-2} \frac{x^{d-4}}{(d-4)!} dx \\
& \quad \quad + \cdots + \int C_3 \frac{x}{1!} dx + \int C_2 dx\\
& \,\, = M   \sum_{i=1}^{k}    m_i  \frac{x^{i+d-1}}{(i+d-1)!/i!}  +  C_{d-1} \frac{x^{d-2}}{(d-2)!}  +  C_{d-2} \frac{x^{d-3}}{(d-3)!} + \cdots +  C_3 \frac{x^2}{2!}  +  C_2 \frac{x}{1!} +C_1\\
\end{align*}	

\begin{align*}
y^{(0)}	& \,\, =  \int y^{(1)} dx\\
& \,\, = \int( M    \sum_{i=1}^{k}    m_i  \frac{x^{i+d-1}}{(i+d-1)!/i!}  +  C_{d-1} \frac{x^{d-2}}{(d-2)!}  +  C_{d-2} \frac{x^{d-3}}{(d-3)!} + \cdots +  C_3 \frac{x^2}{2!}  +  C_2 \frac{x}{1!} +C_1) dx\\
& \,\, =  M    \sum_{i=1}^{k}    m_i  \frac{x^{i+d}}{(i+d)!/i!}  +  C_{d-1} \frac{x^{d-1}}{(d-1)!}  +  C_{d-2} \frac{x^{d-2}}{(d-2)!} + \cdots +  C_3 \frac{x^3}{3!}  +  C_2 \frac{x^2}{2!} +  C_1 \frac{x}{1!} +C_0`\\
\end{align*}	


Applying the initial conditions, we have 

\begin{equation} \label{IVC.111}
\begin{aligned}
	y^{(0)}	& \,\, =  \sum_{i=1}^{k}  \frac{M m_i}{(i+d)!/i!} x^{i+d} + \frac{M_{d-1} }{(d-1)!} x^{d-1}  +  \frac{M_{d-2} }{(d-2)!} x^{d-2} + \cdots +  \frac{M_{2}}{2!} x^2  +   \frac{M_{1}}{1!} x + M_0\\
	& \,\, =  \sum_{i=1}^{k} \frac{i! M m_i}{(i+d)!} x^{i+d} + \frac{M_{d-1} }{(d-1)!} x^{d-1}  +  \frac{M_{d-2} }{(d-2)!} x^{d-2} + \cdots +  \frac{M_{2}}{2!} x^2  +   \frac{M_{1}}{1!} x + M_0.
\end{aligned}	
\end{equation}



\begin{align}\label{sequence112}
\int^{\!^{(d)}}_{\!_{(*)}} (\lambda) 
&=\left\langle1^{ \frac{M_{1} }{1!}  },2^{  \frac{M_{2} }{2!} } ,\dots,(d-1)^{\frac{M_{d-1} }{(d-1)!} }, d^{0}, (d+1)^{\frac{1! M m_1}{(1+d)!}},   \dots, (k+d)^{\frac{k! M m_k}{(k+d)!} } \right\rangle.
\end{align}	

\begin{theorem}\label{thm.p.d.e.1}
Let $\lambda=\langle1^{m_1},2^{m_2},\dots,k^{m_k}\rangle$ be a partition, and let 
\begin{align}
	y^{(d)}	& \,\, =  M f_\lambda(x)= M \sum_{i=1}^{k}  m_i x^{i},
\end{align}	
be a $(\lambda,(*))$-partition differential equation subject to the initial conditions
\begin{equation*}
	(*)   \,\,\, . \,\,\, . \,\,\, . \,\,\, . \,\,\, . \,\,\, . \,\,\, . \,\,\,
	\begin{cases}
		y^{(d-1)}(x_0)	& \,\, = M_{d-1} \\
		y^{(d-2)}(x_0)	& \,\, =  M_{d-2}\\
		\quad \quad .	& \quad \quad . \\
		\quad \quad .	& \quad \quad . \\
		\quad \quad .	& \quad \quad . \\
		y^{(1)}(x_0)	& \,\, =  M_1\\
		y^{(0)}(x_0)	& \,\, = M_0 
	\end{cases}       
\end{equation*}
Then 
\begin{enumerate}[label=(\roman*)]
	\item We have 
	\begin{equation*}	\begin{aligned}\label{sequence11113}
			\left( \int^{\!^{(d)}}_{\!_{(*)}} (\lambda) \right)^{\!^{(1)}}
			&=	 \int^{\!^{(d-1)}}_{\!_{(*)}} (\lambda) \\
		\end{aligned}	
	\end{equation*} 
	\item In general, we have 
	
	\begin{equation}
		\begin{aligned}\label{sequence11114}
			\left( \int^{\!^{(d)}}_{\!_{(*)}} (\lambda) \right)^{\!^{(t)}}
			&= \int^{\!^{(d-t)}}_{\!_{(*)}} (\lambda)\\
		\end{aligned}	
	\end{equation} 
	for all $1 \leq t \leq d$.
\end{enumerate}
\end{theorem}
\begin{proof}
We prove part (ii) (part (i) is a particular of part (ii)). 	Using (\ref{IVC.111}) , we have 
\begin{equation} \label{IVC.112}
	\begin{aligned}
		\frac{d^t}{dx^t} \,\, y^{(0)} & \,\, = 	\frac{d^t}{dx^t} \,\,( \sum_{i=1}^{k} \frac{i! M m_i}{(i+d)!} x^{i+d} + \frac{M_{d-1} }{(d-1)!} x^{d-1}  +  \frac{M_{d-2} }{(d-2)!} x^{d-2} + \cdots +  \frac{M_{2}}{2!} x^2  +   \frac{M_{1}}{1!} x + M_0)\\
		& \,\, =  \sum_{i=1}^{k} \frac{i! M m_i}{(i+d-t)!} x^{i+d-t} + \frac{M_{d-1} }{(d-t-1)!} x^{d-t-1}  +  %\frac{M_{d-2} }{(d-t-2)!} x^{d-t-2}		 + 
		\cdots +  \frac{M_{t+2}}{2!} x^2  +   \frac{M_{t+1}}{1!} x + M_t\\
		& \,\, =  \sum_{i=1}^{k} \frac{i! M m_i}{(i+d-t)!} x^{i+d-t} + \frac{M_{d-1} }{(d-(t+1))!} x^{d-(t+1)} +  %\frac{M_{d-2} }{(d-t-2)!} x^{d-t-2}		 + 
		\cdots +  \frac{M_{t+2}}{2!} x^2  +   \frac{M_{t+1}}{1!} x + M_t\\
		& \,\, =  y^{(t)}\\
	\end{aligned}	
\end{equation}
which is the solution of the $(\lambda,(*))$-partition differential equation 
\begin{align}
	z^{(d-t)} 	& \,\, =  M f_\lambda(x)= M \sum_{i=1}^{k}  m_i x^{i},
\end{align}	
satisfying the initial conditions
\begin{equation*}
	(*)   \,\,\, . \,\,\, . \,\,\, . \,\,\, . \,\,\, . \,\,\, . \,\,\, . \,\,\,
	\begin{cases}
		z^{(d-t-1)}(x_0)	& \,\, = M_{d-1} \\
		z^{(d-t-2)}(x_0)	& \,\, =  M_{d-2}\\
		\quad \quad .	& \quad \quad . \\
		\quad \quad .	& \quad \quad . \\
		\quad \quad .	& \quad \quad . \\
		z^{(1)}(x_0)	& \,\, =  M_{t+1}\\
		z^{(0)}(x_0)	& \,\, = M_t 
	\end{cases}       
\end{equation*}
for $\lambda=\langle1^{m_1},2^{m_2},\dots,k^{m_k}\rangle$, where  $z^{(0)}=y^{(t)}, \, z^{(1)}=y^{(t+1)}, \, \cdots, \, z^{(d-t)}=y^{(d)}$.
Thus,  
\begin{equation}
	\begin{aligned}\label{sequence11114.2}
		\left( \int^{\!^{(d)}}_{\!_{(*)}} (\lambda) \right)^{\!^{(t)}}
		&= \int^{\!^{(d-t)}}_{\!_{(*)}} (\lambda).\\
	\end{aligned}	
\end{equation} 
\end{proof}





%%%%%%%%%%%%%%%%%%%%%%%%%%%%%%%%%%%%%%%%%%%%%%%%%%%%%%%%%%%%%%%%%%%%%%%%%%












\vspace{.4cm}

\vspace*{3mm} 
\begin{flushright}
\begin{minipage}{148mm}\sc\footnotesize
	
	Adnan Hashim Abdulwahid\\
	College of Business, Engineering, and Technology \\
	Texas A$\&$M University--Texarkana\\
	7101 University Ave, Texarkana, TX, 75503,  USA
	
	{\tt \begin{tabular}{lllll}
			{\it E--mail address} :	&  {\color{blue}
				AAbdulwahid@tamut.edu}\\
			&  {\color{blue} adnanalgebra@gmail.com}\\
	\end{tabular} }\vspace*{3mm}
	
	\vspace{.2cm}
\end{minipage}
\end{flushright}









\begin{thebibliography}{5}
\bibitem{Adnan1} Adnan H. Abdulwahid and Elgaddafi Elamami. Bayer Noise Symmetric Functions and
Some Combinatorial Algebraic Structures. JMA (2023)  Volume 46, Pages 115-148. Available Online  \url{https://journals.prz.edu.pl/jma/article/view/1527/1146}.

\bibitem{Adnan2} Adnan H. Abdulwahid. Bayer Noise Quasisymmetric Functions and Some Combinatorial Algebraic Structures. CGASA (2024). Volume 21, Issue 1. Available Online 
\url{https://cgasa.sbu.ac.ir/article_104669_b2dd38f31d3918c0ee50708ef4570204.pdf}. 	

\bibitem{Andrews} George E. Andrews. The Theory of Partitions. Cambridge, England: Cambridge University Press, 1998.

\bibitem{Dawsey} Madeline Locus Dawsey, Tyler Russell, and Dannie Urban. Derivatives and Integrals of Polynomials Associated
with Integer Partitions. {\bf 2022}.
\url{https://browse.arxiv.org/pdf/2108.00943.pdf}


\bibitem{Grinberg} Darij Grinberg. Victor Reiner. Hopf algebras in combinatorics. {\it Lecture notes, Vrije Universiteit Brussel} {\bf 2020}.
\url{https://www.cip.ifi.lmu.de/~grinberg/algebra/HopfComb.pdf}

\bibitem{Macdonald}
Ian Grant Macdonald.
Symmetric functions and Hall polynomials.
2nd edition, Oxford University Press, Oxford-New York, 1995.

\bibitem{Meliot} 
Pierre-Lo\"ic M\'eliot.
Representation Theory
of Symmetric Groups,
{\it Discrete Mathematics and its Applications},
CRC Press 2017.

\bibitem{Mendes}
Anthony Mendes, Jeffrey Remmel,
Counting with Symmetric Functions.
{\it Developments in Mathematics} {\bf 43},
Springer 2015.


\bibitem{Sagan} Bruce E. Sagan.
Combinatorics: The Art of Counting.
Draft of a textbook, 2020.
\url{https://users.math.msu.edu/users/bsagan/Books/Aoc/aoc.pdf}.

\bibitem{Sam-symf}
Steven V. Sam,
Notes for Math 740 (Symmetric Functions),
27 April 2017.
\url{https://www.math.wisc.edu/~svs/740/notes.pdf}

\bibitem{Stanley}
Richard P. Stanley. 
Enumerative Combinatorics, Volumes 1 and 2.
{\it Cambridge Studies in Advanced Mathematics}, {\bf 49} and {\bf 62}.
Cambridge University Press, Cambridge, 2nd edition 2011 (volume 1)
and 1st edition 1999 (volume 2).

\bibitem{Wildon2016} Mark Wildon. An involutive introduction to symmetric functions. 1 July 2017. \url{http://www.ma.rhul.ac.uk/~uvah099/teaching.html}






\end{thebibliography}  

\end{document}
